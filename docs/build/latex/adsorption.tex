%% Generated by Sphinx.
\def\sphinxdocclass{report}
\documentclass[letterpaper,10pt,english]{sphinxmanual}
\ifdefined\pdfpxdimen
   \let\sphinxpxdimen\pdfpxdimen\else\newdimen\sphinxpxdimen
\fi \sphinxpxdimen=.75bp\relax
\ifdefined\pdfimageresolution
    \pdfimageresolution= \numexpr \dimexpr1in\relax/\sphinxpxdimen\relax
\fi
%% let collapsible pdf bookmarks panel have high depth per default
\PassOptionsToPackage{bookmarksdepth=5}{hyperref}

\PassOptionsToPackage{booktabs}{sphinx}
\PassOptionsToPackage{colorrows}{sphinx}

\PassOptionsToPackage{warn}{textcomp}
\usepackage[utf8]{inputenc}
\ifdefined\DeclareUnicodeCharacter
% support both utf8 and utf8x syntaxes
  \ifdefined\DeclareUnicodeCharacterAsOptional
    \def\sphinxDUC#1{\DeclareUnicodeCharacter{"#1}}
  \else
    \let\sphinxDUC\DeclareUnicodeCharacter
  \fi
  \sphinxDUC{00A0}{\nobreakspace}
  \sphinxDUC{2500}{\sphinxunichar{2500}}
  \sphinxDUC{2502}{\sphinxunichar{2502}}
  \sphinxDUC{2514}{\sphinxunichar{2514}}
  \sphinxDUC{251C}{\sphinxunichar{251C}}
  \sphinxDUC{2572}{\textbackslash}
\fi
\usepackage{cmap}
\usepackage[T1]{fontenc}
\usepackage{amsmath,amssymb,amstext}
\usepackage{babel}



\usepackage{tgtermes}
\usepackage{tgheros}
\renewcommand{\ttdefault}{txtt}



\usepackage[Bjarne]{fncychap}
\usepackage{sphinx}

\fvset{fontsize=auto}
\usepackage{geometry}


% Include hyperref last.
\usepackage{hyperref}
% Fix anchor placement for figures with captions.
\usepackage{hypcap}% it must be loaded after hyperref.
% Set up styles of URL: it should be placed after hyperref.
\urlstyle{same}

\addto\captionsenglish{\renewcommand{\contentsname}{Contents:}}

\usepackage{sphinxmessages}
\setcounter{tocdepth}{1}



\title{Adsorption Documentation}
\date{Mar 07, 2024}
\release{0.2.0\sphinxhyphen{}alpha}
\author{Kazem Zhour}
\newcommand{\sphinxlogo}{\vbox{}}
\renewcommand{\releasename}{Release}
\makeindex
\begin{document}

\ifdefined\shorthandoff
  \ifnum\catcode`\=\string=\active\shorthandoff{=}\fi
  \ifnum\catcode`\"=\active\shorthandoff{"}\fi
\fi

\pagestyle{empty}
\sphinxmaketitle
\pagestyle{plain}
\sphinxtableofcontents
\pagestyle{normal}
\phantomsection\label{\detokenize{index::doc}}


\noindent{\hspace*{\fill}\sphinxincludegraphics{{logo}.png}\hspace*{\fill}}

\sphinxAtStartPar
Welcome to the documentation for the Adsorption script. This script facilitates the adsorption of molecules onto surfaces, offering the flexibility to manipulate both the adsorption position and the orientation of the molecule.


\chapter{Contents}
\label{\detokenize{index:contents}}
\sphinxstepscope


\section{Installation}
\label{\detokenize{installation:installation}}\label{\detokenize{installation::doc}}
\sphinxAtStartPar
To use the Adsorption script, follow these steps:
\begin{enumerate}
\sphinxsetlistlabels{\arabic}{enumi}{enumii}{}{.}%
\item {} 
\sphinxAtStartPar
Install Python (if not already installed).

\item {} 
\sphinxAtStartPar
Install the Atomic Simulation Environment (ASE) library.

\item {} 
\sphinxAtStartPar
Make sure that tkinter package is installed for the GUI.

\item {} 
\sphinxAtStartPar
Download the Adsorption.py script from the provided source.

\item {} 
\sphinxAtStartPar
You’re ready to use the script!

\end{enumerate}

\sphinxAtStartPar
You can install the ASE library using pip:
c
.. code\sphinxhyphen{}block:: shell
\begin{quote}

\sphinxAtStartPar
pip install ase
\end{quote}

\sphinxstepscope


\section{Usage}
\label{\detokenize{usage:usage}}\label{\detokenize{usage::doc}}
\sphinxAtStartPar
To use the Adsorption script, run it from the command line with the following syntax:

\begin{sphinxVerbatim}[commandchars=\\\{\}]
python3\PYG{+w}{ }Adsorption.py\PYG{+w}{ }surface\PYGZus{}file.vasp\PYG{+w}{ }molecule\PYGZus{}file.xyz\PYG{+w}{ }\PYGZhy{}\PYGZhy{}origine\PYG{+w}{ }\PYG{l+m}{0}\PYG{+w}{ }\PYGZhy{}\PYGZhy{}vertex\PYG{+w}{ }\PYG{l+m}{1}\PYG{+w}{ }\PYGZhy{}\PYGZhy{}adsorb\PYGZus{}index\PYG{+w}{ }\PYG{l+m}{2}\PYG{+w}{ }\PYGZhy{}\PYGZhy{}height\PYG{+w}{ }\PYG{l+m}{3}.5\PYG{+w}{ }\PYGZhy{}\PYGZhy{}theta\PYGZus{}z\PYG{+w}{ }\PYG{l+m}{30}
\end{sphinxVerbatim}

\sphinxAtStartPar
This command performs adsorption of a molecule on a surface. You need to provide the filenames of the surface and molecule files, as well as the indices of the atoms defining the direction of the molecule. Additionally, specify the index of the atom on the surface for adsorption, the desired adsorption height, and the angle of rotation arround z axis..

\sphinxstepscope


\section{Examples}
\label{\detokenize{examples:examples}}\label{\detokenize{examples::doc}}
\sphinxAtStartPar
Here are some examples demonstrating the usage of the Adsorption script:

\sphinxAtStartPar
Example 1: Perform adsorption of a molecule on a surface

\begin{sphinxVerbatim}[commandchars=\\\{\}]
python3\PYG{+w}{ }Adsorption.py\PYG{+w}{ }surface\PYGZus{}file.vasp\PYG{+w}{ }molecule\PYGZus{}file.xyz\PYG{+w}{ }\PYGZhy{}\PYGZhy{}origine\PYG{+w}{ }\PYG{l+m}{0}\PYG{+w}{ }\PYGZhy{}\PYGZhy{}vertex\PYG{+w}{ }\PYG{l+m}{1}\PYG{+w}{ }\PYGZhy{}\PYGZhy{}adsorb\PYGZus{}index\PYG{+w}{ }\PYG{l+m}{2}\PYG{+w}{ }\PYGZhy{}\PYGZhy{}height\PYG{+w}{ }\PYG{l+m}{3}.5\PYG{+w}{ }\PYGZhy{}\PYGZhy{}theta\PYGZus{}z\PYG{+w}{ }\PYG{l+m}{30}
\end{sphinxVerbatim}

\sphinxAtStartPar
This example performs adsorption of a molecule on a surface using the specified input files and parameters.

\sphinxstepscope


\section{Dependencies}
\label{\detokenize{dependencies:dependencies}}\label{\detokenize{dependencies::doc}}
\sphinxAtStartPar
The Adsorption script has the following dependencies:
\begin{itemize}
\item {} 
\sphinxAtStartPar
Python

\item {} 
\sphinxAtStartPar
Atomic Simulation Environment (ASE) library

\end{itemize}

\sphinxAtStartPar
The ASE library is required for reading and writing structure files. It supports various file formats such as POSCAR, xyz, and cif. You can install the ASE library using pip:

\begin{sphinxVerbatim}[commandchars=\\\{\}]
pip\PYG{+w}{ }install\PYG{+w}{ }ase
\end{sphinxVerbatim}

\sphinxstepscope


\section{Module Documentation}
\label{\detokenize{module_documentation:module-documentation}}\label{\detokenize{module_documentation::doc}}
\sphinxAtStartPar
Here, we provide documentation for the Python module \sphinxtitleref{Adsorption}.
\index{module@\spxentry{module}!Adsorption@\spxentry{Adsorption}}\index{Adsorption@\spxentry{Adsorption}!module@\spxentry{module}}\phantomsection\label{\detokenize{module_documentation:module-Adsorption}}
\sphinxAtStartPar
Author: Kazem Zhour

\sphinxAtStartPar
Date: 26.04.2024
\begin{description}
\sphinxlineitem{Description: Adsorption of molecules on surfaces.}
\sphinxAtStartPar
This Python script facilitates the adsorption of molecules onto surfaces, offering the flexibility to
manipulate both the adsorption position and the orientation of the molecule.

\sphinxlineitem{Usage: This script requires input files for the surface and the molecule. The user will be prompted }
\sphinxAtStartPar
to enter the filenames for these input files. Additionally, the script prompts the user to 
specify the origin and head atoms for defining the direction along the molecule, the angle 
for rotating the molecule around the z\sphinxhyphen{}axis, the atom of the slab on which to adsorb the 
molecule, and the desired adsorption height.

\sphinxlineitem{Dependencies: This script requires the Atomic Simulation Environment (ASE) library for reading and }
\sphinxAtStartPar
writing structure files. ASE can handle various file formats such as POSCAR, xyz, and cif.

\sphinxlineitem{Example:}
\sphinxAtStartPar
python3 Adsorption.py surface\_file.vasp molecule\_file.xyz \textendash{}origine 0 \textendash{}vertex 1 \textendash{}adsorb\_index 2 \textendash{}height 3.5

\sphinxAtStartPar
This example demonstrates how to use the script to perform adsorption of a molecule on a surface. 
The user is prompted to input the filenames of the surface and molecule files, as well as the indices 
of the atoms defining the direction of the molecule. Then, the user specifies the index of the atom on 
the surface for adsorption and the desired adsorption height. The script generates an output containing 
the combined structure of the surface and the adsorbed molecule.

\end{description}
\index{calculate\_molecule\_orientation() (in module Adsorption)@\spxentry{calculate\_molecule\_orientation()}\spxextra{in module Adsorption}}

\begin{fulllineitems}
\phantomsection\label{\detokenize{module_documentation:Adsorption.calculate_molecule_orientation}}
\pysigstartsignatures
\pysiglinewithargsret{\sphinxcode{\sphinxupquote{Adsorption.}}\sphinxbfcode{\sphinxupquote{calculate\_molecule\_orientation}}}{\sphinxparam{\DUrole{n}{molecule}}\sphinxparamcomma \sphinxparam{\DUrole{n}{origine}}\sphinxparamcomma \sphinxparam{\DUrole{n}{vertex}}}{}
\pysigstopsignatures
\sphinxAtStartPar
Calculates the orientation of the molecule based on user input.
\begin{description}
\sphinxlineitem{Args:}
\sphinxAtStartPar
molecule (ASE Atoms object): The molecule structure.

\sphinxlineitem{Returns:}
\sphinxAtStartPar
phi (float): Azimuthal angle.
theta (float): Polar angle.

\end{description}

\end{fulllineitems}

\index{main() (in module Adsorption)@\spxentry{main()}\spxextra{in module Adsorption}}

\begin{fulllineitems}
\phantomsection\label{\detokenize{module_documentation:Adsorption.main}}
\pysigstartsignatures
\pysiglinewithargsret{\sphinxcode{\sphinxupquote{Adsorption.}}\sphinxbfcode{\sphinxupquote{main}}}{}{}
\pysigstopsignatures
\sphinxAtStartPar
Main function to perform adsorption of molecules on surfaces.

\end{fulllineitems}

\index{read\_input\_files() (in module Adsorption)@\spxentry{read\_input\_files()}\spxextra{in module Adsorption}}

\begin{fulllineitems}
\phantomsection\label{\detokenize{module_documentation:Adsorption.read_input_files}}
\pysigstartsignatures
\pysiglinewithargsret{\sphinxcode{\sphinxupquote{Adsorption.}}\sphinxbfcode{\sphinxupquote{read\_input\_files}}}{\sphinxparam{\DUrole{n}{surface\_file}}\sphinxparamcomma \sphinxparam{\DUrole{n}{molecule\_file}}}{}
\pysigstopsignatures
\sphinxAtStartPar
Reads input files for surface and molecule.
\begin{description}
\sphinxlineitem{Returns:}
\sphinxAtStartPar
surface (ASE Atoms object): The surface structure.
molecule (ASE Atoms object): The molecule structure.

\end{description}

\end{fulllineitems}

\index{rotate\_molecule() (in module Adsorption)@\spxentry{rotate\_molecule()}\spxextra{in module Adsorption}}

\begin{fulllineitems}
\phantomsection\label{\detokenize{module_documentation:Adsorption.rotate_molecule}}
\pysigstartsignatures
\pysiglinewithargsret{\sphinxcode{\sphinxupquote{Adsorption.}}\sphinxbfcode{\sphinxupquote{rotate\_molecule}}}{\sphinxparam{\DUrole{n}{molecule}}\sphinxparamcomma \sphinxparam{\DUrole{n}{phi}}\sphinxparamcomma \sphinxparam{\DUrole{n}{theta}}}{}
\pysigstopsignatures
\sphinxAtStartPar
Rotates the molecule to align with the calculated orientation.
\begin{description}
\sphinxlineitem{Args:}
\sphinxAtStartPar
molecule (ASE Atoms object): The molecule structure.
phi (float): Azimuthal angle.
theta (float): Polar angle.

\end{description}

\end{fulllineitems}

\index{translate\_molecule() (in module Adsorption)@\spxentry{translate\_molecule()}\spxextra{in module Adsorption}}

\begin{fulllineitems}
\phantomsection\label{\detokenize{module_documentation:Adsorption.translate_molecule}}
\pysigstartsignatures
\pysiglinewithargsret{\sphinxcode{\sphinxupquote{Adsorption.}}\sphinxbfcode{\sphinxupquote{translate\_molecule}}}{\sphinxparam{\DUrole{n}{molecule}}\sphinxparamcomma \sphinxparam{\DUrole{n}{surface}}\sphinxparamcomma \sphinxparam{\DUrole{n}{adsorb\_index}}\sphinxparamcomma \sphinxparam{\DUrole{n}{height}}}{}
\pysigstopsignatures
\sphinxAtStartPar
Translates the molecule to the adsorption site on the surface.
\begin{description}
\sphinxlineitem{Args:}
\sphinxAtStartPar
molecule (ASE Atoms object): The molecule structure.
surface (ASE Atoms object): The surface structure.
adsorb\_index (int): Index of the adsorption site atom in the surface.
height (float): Height of adsorption.

\end{description}

\end{fulllineitems}


\sphinxAtStartPar
Below are some key functions provided by the \sphinxtitleref{Adsorption} module:
\index{read\_input\_files() (in module Adsorption)@\spxentry{read\_input\_files()}\spxextra{in module Adsorption}}

\begin{fulllineitems}
\phantomsection\label{\detokenize{module_documentation:id0}}
\pysigstartsignatures
\pysiglinewithargsret{\sphinxcode{\sphinxupquote{Adsorption.}}\sphinxbfcode{\sphinxupquote{read\_input\_files}}}{\sphinxparam{\DUrole{n}{surface\_file}}\sphinxparamcomma \sphinxparam{\DUrole{n}{molecule\_file}}}{}
\pysigstopsignatures
\sphinxAtStartPar
Reads input files for surface and molecule.
\begin{description}
\sphinxlineitem{Returns:}
\sphinxAtStartPar
surface (ASE Atoms object): The surface structure.
molecule (ASE Atoms object): The molecule structure.

\end{description}

\end{fulllineitems}

\index{calculate\_molecule\_orientation() (in module Adsorption)@\spxentry{calculate\_molecule\_orientation()}\spxextra{in module Adsorption}}

\begin{fulllineitems}
\phantomsection\label{\detokenize{module_documentation:id1}}
\pysigstartsignatures
\pysiglinewithargsret{\sphinxcode{\sphinxupquote{Adsorption.}}\sphinxbfcode{\sphinxupquote{calculate\_molecule\_orientation}}}{\sphinxparam{\DUrole{n}{molecule}}\sphinxparamcomma \sphinxparam{\DUrole{n}{origine}}\sphinxparamcomma \sphinxparam{\DUrole{n}{vertex}}}{}
\pysigstopsignatures
\sphinxAtStartPar
Calculates the orientation of the molecule based on user input.
\begin{description}
\sphinxlineitem{Args:}
\sphinxAtStartPar
molecule (ASE Atoms object): The molecule structure.

\sphinxlineitem{Returns:}
\sphinxAtStartPar
phi (float): Azimuthal angle.
theta (float): Polar angle.

\end{description}

\end{fulllineitems}


\sphinxAtStartPar
You can add more functions similarly.


\chapter{Indices and Search}
\label{\detokenize{index:indices-and-search}}\begin{itemize}
\item {} 
\sphinxAtStartPar
\DUrole{xref,std,std-ref}{genindex}

\item {} 
\sphinxAtStartPar
\DUrole{xref,std,std-ref}{modindex}

\item {} 
\sphinxAtStartPar
\DUrole{xref,std,std-ref}{search}

\end{itemize}


\renewcommand{\indexname}{Python Module Index}
\begin{sphinxtheindex}
\let\bigletter\sphinxstyleindexlettergroup
\bigletter{a}
\item\relax\sphinxstyleindexentry{Adsorption}\sphinxstyleindexpageref{module_documentation:\detokenize{module-Adsorption}}
\end{sphinxtheindex}

\renewcommand{\indexname}{Index}
\printindex
\end{document}